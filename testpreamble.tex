\usepackage[left=0.15in,right=0.5in,top=0.5in,bottom=0.75in]{geometry} % never use the anysize package!
\usepackage{graphicx}		% de facto new standard for graphics
\usepackage{xcolor}
\usepackage{multicol,multirow}
\usepackage{array}	
\usepackage{comment}
%\usepackage{sgame}
\usepackage{tikz}
\usepackage{etoolbox}
\usepackage{hyperref}
\usepackage{marvosym}
\usepackage{tikz}
\usetikzlibrary{calc}
\usepackage{amsmath,amssymb}
\usepackage{rotate}
\usepackage{setspace}
\usepackage{xintexpr}
%\usepackage{tabularx} % cannot use tabularx; may consider booktabs.

%\setlength{\parindent}{0in}	% easier than putting \noindent at every paragraph

% basic shortcuts:
\newcommand{\+}{\item}			% easier than typing \item a lot
\newcommand{\bi}{\begin{itemize}}
\newcommand{\ei}{\end{itemize}}
\newcommand{\bd}{\begin{description}}
\newcommand{\ed}{\end{description}}
\newcommand{\be}{\begin{enumerate}}
\newcommand{\ee}{\end{enumerate}}
\newcommand{\rr}{\raggedright}

% some hyperlink and table formatting shortcuts:
\renewcommand{\thefootnote}{\fnsymbol{footnote}} 	
\definecolor{webblue}{rgb}{0,0,0.75}
\definecolor{rightgreen}{rgb}{0,0.65,0}
\definecolor{footlineblue}{rgb}{0.392,0.706,0.941}
\newcommand{\sep}{1mm}
\newcommand{\negsep}{-4mm}

% formdefs is the specific formatting information required for the 
% Scantron form you wish to use. 
\newcommand{\formdefs}{old4521.txt}


%%% Stuff to load up before getting going with the rest of the test:

% counter for the questions:
\newcounter{saveenum}
% And also create a counter for the within-question answers. 
% IMPORTANT NOTE:
% the "answer" counter resets to zero when you increment the "saveenum" counter.
\newcounter{answer}[saveenum] 

\newif\ifkey
\keyfalse


% tabular formatting and shortcuts for the questions.
% Typically you won't care about any of these except the \question macro,
% Which is the idea: it makes it clean and easy to format each question. 

% How wide will the text fields be for questions and answers?
\newcommand{\howwide}{6.7in}
\newcommand{\columnonewidth}{0.13in}
\newcommand{\columntwowidth}{0.17in}
\newcommand{\graphicquestionwidth}{4.0in}
\newcommand{\graphiccolumnwidth}{3.0in}

% create the tabular environment for this question:
\newcommand{\qtab}{\begin{tabular}{b{\columnonewidth} b{\columntwowidth} p{\howwide}}}

% print the question number, in bold, and increment to the next tabular column.
\newcommand{\mft}{\textbf{\arabic{saveenum}}. & }

% if you need to have "question 4a", use this macro instead:
\newcommand{\mfta}[1]{\textbf{\arabic{saveenum}#1}. & }

% span two columns with the string input by the user (this is the question block)
\newcommand{\mcp}[1]{\multicolumn{2}{p{\howwide}}{#1}  \\[\sep]}

% Tie all these smaller macros together into a simple 
% \question{Question text goes here},
% though note you still have to close out the tabular environment 
% with \eqq or \eqqnoskip, below.
% The \question macro also increments the question counter, and 

\newcommand{\question}[1]{ \stepcounter{saveenum} \stepcounter{answer} \qtab\mft\mcp{#1}}

% Usually you want some vertical space between questions. 
% The \eqq macro closes out the tabular environment and adds a bigskip.
% The \eqnoskip macro closes out the tabular environment with no extra space.
\newcommand{\eqq}{\end{tabular} \bigskip }
\newcommand{\eqnoskip}{\end{tabular} }

% this macro neatly formats each multiple choice answer and increments the 
% within-question answer counter:
\newcommand{\ans}[1]{ & (\alph{answer}) & #1 \stepcounter{answer} \\ } 
	% both \addtocounter{answer}{1} and \stepcounter{answer} add a line feed 
	% UNLESS you put the stepcounter statement before the end of the tabular line break!

% this macro lets you flag the correct answer in the LaTeX source, but only
% have it show up when you specifically change the "key" variable from false to true:
\newcommand{\rightans}[1]{ 
\ifkey
	\immediate\write\tempfile{\arabic{answer}}
	
	& ({\color{rightgreen}\textbf{\alph{answer}}}) & \textbf{#1} \stepcounter{answer} \\ 

\else
	& (\alph{answer}) & #1 \stepcounter{answer} \\ 
% both \addtocounter{answer}{1} and \stepcounter{answer} add a line feed 
% UNLESS you put the stepcounter statement before the end of the tabular line break!
\fi
}

% if you just want a checkbox (say, for a Likert-scale or yes/no question):
% NOTE: this ends the line after each question.
\newcommand{\checkbox}[1]{& $\Box$ & #1 \\[\sep]}

% Some customization for, say, extra credit questions:
\newcommand{\fullwidth}[1] {\multicolumn{3}{p{\howwide}}{#1 \\ }  }
\newcommand{\fulltextwidth}[1]{& \mcp{#1}}

%%%%%%%%%%%%%%%%%%%%%%%%%%%%%%%%%%%%%%%%%%%%%%%%%%%%%%%%%%%%%%%%%%%%%%%%%%%%%%%%%%%%%%
% using the \xint package to do some math:
%%%%%%%%%%%%%%%%%%%%%%%%%%%%%%%%%%%%%%%%%%%%%%%%%%%%%%%%%%%%%%%%%%%%%%%%%%%%%%%%%%%%%%
\def\pagewcomputed #1{\xinttheexpr round(\oneover{#1} - 0.08,2)\relax}

%%%%%%%%%%%%%%%%%%%%%%%%%%%%%%%%%%%%%%%%%%%%%%%%%%%%%%%%%%%%%%%%%%%%%%%%%%%%%%%%%%%%%%
% Likert scale question for survey-type "tests":
% argument 1: multiplier for column count
% argument 2: columns for the question to span
% argument 3: question text
%%%%%%%%%%%%%%%%%%%%%%%%%%%%%%%%%%%%%%%%%%%%%%%%%%%%%%%%%%%%%%%%%%%%%%%%%%%%%%%%%%%%%%

\newcommand{\likertq}[3]{

\def\pagewidthmultiplier{0.19} % assumes five answers; if there are not...
\xintNewFloatExpr{\oneover}[1]{1/#1}

\stepcounter{saveenum} \stepcounter{answer}
\def\columncount{#1}
\def\pwc{\pagewcomputed{\columncount}}

\begin{tabular}{b{\columnonewidth} *{\columncount}{ l p{\pwc\linewidth} } }
\mft \multicolumn{#2}{p{\linewidth}}{\parbox[t]{6.5in}{#3}}\\[\sep]
}


% checkbox, column break, answer text:
\newcommand{\loption}[1]{& $\Box$ & #1}

% the usual likert scales for agreement with a given statement:
\newcommand{\agreefive}{
\loption{Strongly
\newline Disagree}
\loption{Disagree}
\loption{I have 
\newline no opinion}
\loption{Agree}
\loption{Strongly
\newline Agree}
}

\newcommand{\yesno}{
\loption{Yes}
\loption{No}
& & 
& & 
& & 
}

\newcommand{\truefalse}{
\loption{True}
\loption{False}
& & 
& & 
& & 
}

\newcommand{\truefalseq}[1]{
\likertq{5}{10}{#1}
\truefalse{}
\\[\sep]
\eqq
}

% bog standard likert scale question:
\newcommand{\likertqfive}[1]{
\likertq{5}{10}{#1}
\agreefive{}
\\[\sep]
\eqq
}



% the less common seven-stage (and not that awesome) likert scales for agreement with a given statement:
\newcommand{\agreeseven}{\loption{Strongly Disagree} \loption{Disagree} \loption{Weakly Disagree} \loption{I have no opinion} \loption{Weakly Agree}  \loption{Agree} \loption{Strongly Agree}
}


% macro to put some text on what is roughly the footnote line, 
% at least for this exam's layout (you should compile more than once):
\newcommand{\footlineextra}[1]{
    \begin{tikzpicture}[remember picture,overlay]
        \node[yshift=4ex,xshift=8ex,anchor=south west] at (current page.south west) 
        	{\scriptsize \hspace{0ex}{\color{black}#1}};
    \end{tikzpicture}
}


% Fuzzy stuff, purely for personal preference.
% --------------------------------------------
\hfuzz=12pt % Don't bother to report over-full boxes < 12pt
\vfuzz=12pt % Don't bother to report over-full boxes < 12pt
\tolerance=500 %

% This macro uses TikZ to put a letter in the upper left corner of the exam, 
% independent of any margins or other formatting:
\newcommand{\testversionmarker}[1]{% Marker for test version "A" or "B"
\begin{tikzpicture}[remember picture,overlay]
	\node[yshift=-0.35in, xshift=0.35in, anchor=north west]
	at (current page.north west)
	{Test Version #1};
\end{tikzpicture}}

\newcommand{\upperleftcomment}[1]{% arbitrary text
\begin{tikzpicture}[remember picture,overlay]
	\node[yshift=-0.35in, xshift=0.35in, anchor=north west]
	at (current page.north west)
	{#1};
\end{tikzpicture}}


% Make the test header on the test's title page:
\newcommand{\testheader}[4]{\begin{center}% \vspace doesn't work at the top of the page:
{\LARGE \textbf{~\\~\\~\\#1 Exam, #2, #3 }}\\


\vspace{#4}

\ifkey
	~
\else
	{\LARGE Read all instructions carefully.}
	\vspace{#4}
\fi

\end{center}
}

% macro to print some instructions to students on the title page.
% Obviously feel free to change these entries!
\newcommand{\printinstructions}{ 
\ifkey
	\printkey
\else
	\begin{center}
	\begin{minipage}[c]{6.5in}{
	\Large
\be
%\+ \textbf{Do not print your name anywhere on the test.} 
%\+ \emph{If you print your name on the test, I will take off 10 points.}
%\+ This test should take no more than 60 minutes.
%\+ Each question is worth 2 points.
%\+ If you need more room to write, use the back of the page, and mark the number and letter (if any) of the question you're continuing to work on.
\+ No hats; no headphones.
\+ Power down your cell phone and other devices. 
\+ \textbf{If we see an electronic device, you get a zero.}
% \+ Put all electronic devices under your desk, but able to be seen.
%\+ Phones and tablets may also be put on the instructor's desk, or on the whiteboard marker tray.
\+ Read all the answers for each multiple choice question!
\+ Do not open the test until instructed to do so.
\+ If you leave for any reason, you can't come back in.
\+ This test contains extra credit questions.
%\+ Fill in your last and first names the bubble sheet. 
\+ Print your name \& student ID number on the bubble sheet.
\+ \textbf{Last (family) name goes first on the bubble sheet.}
\+ Fill in the optical marks corresponding to your name and ID number.
%\+ You do not need to fill in your ID number.
\+ Write your name and ID number on the extra credit sheet.
\+ Turn in the whole exam, with the last page attached.
\+ If you get a few of the same answers in a row (say, three ``e" or three ``d''), don't freak out. That happens sometimes---the questions and some of the answers are ordered randomly. Trust yourself.
\+ When you turn in the exam, present your student ID or driver's license to a TA for verification
%\+ At least one of the answers can be found elsewhere in the test.
%\+ I enjoyed teaching you. Apply what you've learned outside the classroom.
\ee
}
	\end{minipage}
	\end{center}
\fi
}

% Simple macro for printing "KEY" on the title page (instead of the instructions)
\newcommand{\printkey}{\begin{center}\begin{LARGE}\textbf{KEY}\end{LARGE}\end{center}}

% Making a slightly custom test section title (includes the opportunity to say how many points each question is worth, plus room for "except as noted" or some other boilerplate)
\newcommand{\sectiontitle}[3]{\begin{center}
\begin{Large}
\textbf{#1}
\end{Large}

\newif\ifsingle
\ifnum#2=1 \singletrue \else \singlefalse \fi

\ifsingle
	\emph{1 point each{}#3.}%,  except as noted}
\else
	\emph{#2 points each{}#3.}%,  except as noted}
\fi
\end{center}
}

% macro to write an inline "______" of user-defined length on the test:
\newcommand{\makeblank}[1]{
\underline{\makebox[#1][l]{}}
}

% macro to write a "Name: ______" on the test:
\newcommand{\addnameline}{\vspace{0.125in}

\begin{tabular}{ll}
\large Name: & \multicolumn{1}{p{5.5in}}{\hrule} \\
\end{tabular}
\vspace{0.125in}
}

%%%%%%%%%%%%%%%%%%%%%%%%%%%%%%%%%%%%%%%%%%%%%%%%%%%%%%%%%%%%%%%%%%%%%%%%%%%%%%%%%
% A little \LaTeX magic to ensure the last page is odd numbered so students can 
% tear it off without losing any content elsewhere on the test:
%
% Note: this bit of code was originally intended to be at the end of a multicols*
% environment. If you need for it to be at the end of a multicols* environment,
% you will have to adjust the code manually for now.
%  
%%%%%%%%%%%%%%%%%%%%%%%%%%%%%%%%%%%%%%%%%%%%%%%%%%%%%%%%%%%%%%%%%%%%%%%%%%%%%%%%%
\newcommand{\ifoddleaveblank}{

\ifodd \value{page}
	\newpage

	\begin{center}
	~
	\vfill
%	\emph{This page (\arabic{page}) intentionally left blank.}
	\emph{This page intentionally left blank.}
	\end{center}%
	\newpage
\else
%	Are you sure this is an even-numbered page?
%	\hbox{}
%	\newpage

\fi
}

% DO print the test if this is a key, or DON'T print the text and
% instead vspace a given amount (to leave room for short answer questions).
\newcommand{\printifkey}[2]{
\ifkey 
	\vspace{0.1in}
	\noindent\textbf{#2} % \\
	\bigskip

\else
	\vspace{#1}
\fi
}

% macro to open up the temporary file for storing answer values:
\newcommand{\opentempfile}[1]{

\ifkey 
	\newwrite\tempfile
	\immediate\openout\tempfile=#1
\fi
}

% YOU CAN SET the output format for the scantron. It will default to
% "old4521.txt" both in the script and as I explicitly specify it here.

% macro to close the temporary file for storing answer values and run the python script:
\newcommand{\closetempfile}[1]{
\ifkey 

	\immediate\closeout\tempfile
%	\immediate\write18{/bin/cat #1 | /usr/bin/perl -e 'while(<STDIN>) {chomp; print "$_,"}' > answerlist1.txt }
	\immediate\write18{/usr/bin/python makesvg.py -o -i "\formdefs"  -f \jobname} % -o means "just print the dots"
\fi
}



% Template for having a graphic in a question:
\newcommand{\graphicquestion}[1]{
\stepcounter{saveenum} \stepcounter{answer} 
\begin{tabular}{b{\graphicquestionwidth}b{\graphiccolumnwidth}}
	\begin{tabular}{b{\columnonewidth} b{\columntwowidth} p{3.25in} }
	\textbf{\arabic{saveenum}}. & 
	\multicolumn{2}{p{\graphicquestionwidth}}{#1} \\[\sep] 
}

\newcommand{\addquestiongraphic}[2]{
\eqnoskip
&
\begin{minipage}[c]{\graphiccolumnwidth}
\begin{center}
\includegraphics[scale=#1, keepaspectratio=true]{#2}\\
\end{center}
\end{minipage}
\\
}

%<EOF>